\documentclass{article}

\usepackage{amsmath}
\usepackage{amsthm}
\usepackage{amssymb}

\begin{document}

\title{Fourier Optics}
\author{Shi-Ning Sun}
\maketitle

\section{Fourier Transform}

\subsection{Definition}

In optics, the Fourier transform is defined as

\begin{align}
    G(f_x, f_y) = \mathcal{F}\{g(x, y)\} = \iint g(x, y) e^{-2\pi j(f_x x + f_y y)} dx dy
\end{align}
and the inverse Fourier transform is defined as
\begin{align}
    g(x, y) = \mathcal{F}^{-1}\{G(f_x, f_y)\} = \iint G(f_x, f_y) e^{2\pi j(f_x x + f_y y)} dx dy
\end{align} 
where both integrations extend from $-\infty$ to $\infty$.

\subsection{Common Fourier Transform Pairs}

$$
\begin{array}{c|c}
    g(x, y) & G(f_x, f_y) \\
    \hline
    e^{-\pi a x^2} & \frac{e^{-\pi f_x^2 / a}}{\sqrt{a}}
\end{array}
$$

To verify Fourier transform formulas in Mathematica, we should do

\begin{verbatim}
FourierTransform[<g>, {x, y}, {fx, fy}, FourierParameters -> {0, -2 Pi}]
\end{verbatim}


\section{Scalar Diffraction}

\subsection{Rayleigh-Sommerfeld Approximation}

By the Rayleigh-Sommerfeld apprxomation, the field on the observation plane is

\begin{align}
    U_2(x, y) = \frac{z}{j\lambda} \iint U_1(\xi, \eta) \frac{e^{jk\sqrt{z^2 + (x-\xi)^2 + (y-\eta)^2}}}
    {z^2 + (x-\xi)^2 + (y-\eta)^2} d\xi d\eta
\end{align}

\subsection{Fresnel Approximation}

When
\begin{align}
    z^2 \gg (x-\xi)^2 + (y-\eta)^2
\end{align}
we can approximate the exponent using the expression $\sqrt{1 + a} \approx 1 + a/2$ where 
$a = [(x-\xi)^2 + (y-\eta)^2]/z^2$, and the denominator by $z^2 + (x-\xi)^2 + (y-\eta)^2 \approx z^2$, 
in which case the expression for the phasor on the observation plane is
\begin{align}
    U_2(x, y) = \frac{e^{jkz}}{j\lambda z} \iint U_1(\xi,\eta) e^{j\frac{k}{2z}((x-\xi)^2 + (y-\eta)^2)} d\xi d\eta
\end{align}
This expression can be regarded as a convolution $U_2 = U_1 * h$, where $h$ is
\begin{align}
    h(x, y) = \frac{e^{jkz}}{j\lambda z} e^{j\frac{k}{2z}(x^2 + y^2)}
\end{align}
By the convolution property for Fourier transform, we can calculate $U_2(x, y)$ by
\begin{align}
    U_2(x, y) = \mathcal{F}^{-1}\{ \mathcal{F}\{U_1(x, y)\} H(f_x, f_y) \},
\end{align}
where $H(f_x, f_y)$ is the transfer function, which is the Fourier transform of $h(x, y)$ and has the expression
\begin{align}
    H(f_x, f_y) = e^{jkz} e^{-j\pi\lambda z(f_x^2 + f_y^2)}
\end{align}

We can also expand the quadratic function on the exponential and pull out the term that do not depend on the intergration variables, in which case
\begin{align}
    U_2(x, y) = \frac{e^{jkz}}{j\lambda z} e^{j\frac{k}{2z}(x^2+y^2)} \iint U_1(\xi,\eta) e^{j\frac{k}{2z}(\eta^2+\xi^2)} e^{j\frac{k}{2z}(x\xi + y\eta)} d\xi d\eta
\end{align}
This integral can be interpreted as the the Fourier transform of the source field times a chirp function.

\subsection{Fraunhofer Approximation}

In far field $z \gg \xi^2 + \eta^2$, we can approximate the chirp term $e^{j\frac{k}{2z}(\xi^2 + \eta^2)}$ 
in the Fresnel approximation as unity. The phasor on the observation plane is

\begin{align}
    U_2(x, y) = \frac{e^{jkz}}{j\lambda z} e^{j\frac{k}{2z}(x^2+y^2)} 
    \iint U_1(\xi,\eta) e^{-j\frac{2\pi}{\lambda z}(x\xi + y\eta)}d\xi d\eta
\end{align}

\end{document}
