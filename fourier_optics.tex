\documentclass{revtex4-2}

\usepackage{amsmath}
\usepackage{amsthm}
\usepackage{amssymb}
\usepackage{minted}

\begin{document}

\title{Fourier Optics}
\author{Shi-Ning Sun}
\maketitle

\section{Fourier Transform}

\subsection{2D Fourier Transform}

In optics, the Fourier transform is defined as
\begin{align}
    G(f_x, f_y) = \mathcal{F}\{g(x, y)\} = \iint g(x, y) e^{-2\pi j(f_x x + f_y y)} dx dy
\end{align}
and the inverse Fourier transform is defined as
\begin{align}
    g(x, y) = \mathcal{F}^{-1}\{G(f_x, f_y)\} = \iint G(f_x, f_y) e^{2\pi j(f_x x + f_y y)} dx dy
\end{align} 
where both integrations extend from $-\infty$ to $\infty$.

\subsection{Common Fourier Transform Pairs}

$$
\begin{array}{c|c}
    g(x, y) & G(f_x, f_y) \\
    \hline
    e^{-\pi a x^2} & \frac{e^{-\pi f_x^2 / a}}{\sqrt{a}}
\end{array}
$$

To verify Fourier transform formulas in Mathematica, we should do

\begin{minted}{mathematica}
FourierTransform[<g>, {x, y}, {fx, fy}, FourierParameters -> {0, -2 Pi}]
\end{minted}

\subsection{MATLAB Example}

Typically, to do a 2D Fourier transform in MATLAB, we can use the {\tt fft2} function.
\begin{minted}{matlab}
    % Spatial domain
    x = -L/2:dx:L/2-dx;
    y = x;
    [X,Y] = meshgrid(x,y);
    g = <obtain-g-from-XY>;

    % Frequency domain
    fx = -1(2*dx):1/L:1/(2*dx)-1/L;
    fy = fx;
    g0 = fftshift(g);
    G0 = fft2(g0)*dx^2; % dx^2 for normalization
    G = fftshift(G0);
\end{minted}

\section{Scalar Diffraction}

\subsection{Optical Field}
An optical field can be written as
\begin{align}
    u(\mathbf{r}, t) = A(\mathbf{r})e^{j(\phi(\mathbf{r}) - \omega t)}
\end{align}
For a plane wave, $A(\mathbf{r}) = A$ and $\phi(\mathbf{r}) = \mathbf{k}\cdot \mathbf{r}$. 
We usually assume the plane wave propagates in the $z$-direction, in which case the expression
for the field is
\begin{align}
    u(\mathbf{r}, t) = A e^{j(kz - \omega t)}
\end{align}

If the field propagates in a linear medium, it is not necessary to carry the temporal term, in which case we
can write down the phasor form of the field as
\begin{align}
    U(\mathbf{r}) = A(\mathbf{r})e^{j\phi(\mathbf{r})}.
\end{align}
Furthermore, if $z$ is the fundamental direction of propagation, then we can write the propagation field as
\begin{align}
    U(x, y) = A(x, y)e^{j\phi(x, y)}.
\end{align}


\subsection{Rayleigh-Sommerfeld Approximation}

By the Rayleigh-Sommerfeld apprxomation, the field on the observation plane is

\begin{align}
    U_2(x, y) = \frac{z}{j\lambda} \iint U_1(\xi, \eta) \frac{e^{jk\sqrt{z^2 + (x-\xi)^2 + (y-\eta)^2}}}
    {z^2 + (x-\xi)^2 + (y-\eta)^2} d\xi d\eta
\end{align}

\subsection{Fresnel Approximation}

When
\begin{align}
    z^2 \gg (x-\xi)^2 + (y-\eta)^2
\end{align}
we can approximate the exponent using the expression $\sqrt{1 + a} \approx 1 + a/2$ where 
$a = [(x-\xi)^2 + (y-\eta)^2]/z^2$, and the denominator by $z^2 + (x-\xi)^2 + (y-\eta)^2 \approx z^2$, 
in which case the expression for the phasor on the observation plane is
\begin{align}
    U_2(x, y) = \frac{e^{jkz}}{j\lambda z} \iint U_1(\xi,\eta) e^{j\frac{k}{2z}((x-\xi)^2 + (y-\eta)^2)} d\xi d\eta
\end{align}
This expression can be regarded as a convolution $U_2 = U_1 * h$, where $h$ is
\begin{align}
    h(x, y) = \frac{e^{jkz}}{j\lambda z} e^{j\frac{k}{2z}(x^2 + y^2)}
\end{align}
By the convolution property of Fourier transform, we can calculate $U_2(x, y)$ by
\begin{align}
    \boxed{ U_2(x, y) = \mathcal{F}^{-1}\{ \mathcal{F}\{U_1(x, y)\} \mathcal{F}\{h(x, y)\} \}. }
\end{align}
The Fourier transform of $h(x, y)$ has an analytical expression is the transfer function $H(f_x, f_y)$,
which has the expression
\begin{align}
    H(f_x, f_y) = e^{jkz} e^{-j\pi\lambda z(f_x^2 + f_y^2)}
\end{align}
so the observation plane field can be alternatively calculated as
\begin{align}
    \boxed{ U_2(x, y) = \mathcal{F}^{-1}\{ \mathcal{F}\{U_1(x, y)\} H(f_x, f_y) \}. }
\end{align}

We can also expand the quadratic function on the exponential and pull out the term that do not
depend on the intergration variables, in which case
\begin{align}
    U_2(x, y) = \frac{e^{jkz}}{j\lambda z} e^{j\frac{k}{2z}(x^2+y^2)} 
    \iint U_1(\xi,\eta) e^{j\frac{k}{2z}(\eta^2+\xi^2)} e^{j\frac{k}{2z}(x\xi + y\eta)} d\xi d\eta
\end{align}
This integral can be interpreted as the the Fourier transform of the source field times a chirp function.

\subsection{Fraunhofer Approximation}

In far field $z \gg \xi^2 + \eta^2$, we can approximate the chirp term $e^{j\frac{k}{2z}(\xi^2 + \eta^2)}$ 
in the Fresnel approximation as unity. The phasor on the observation plane is

\begin{align}
    U_2(x, y) = \frac{e^{jkz}}{j\lambda z} e^{j\frac{k}{2z}(x^2+y^2)} 
    \iint U_1(\xi,\eta) e^{-j\frac{2\pi}{\lambda z}(x\xi + y\eta)}d\xi d\eta
\end{align}

The integral can be regarded as as Fourier transform of the source field with the frequency variables
\begin{align}
    f_\xi = \frac{x}{\lambda z}, \ f_\eta = \frac{y}{\lambda z} \iff
    x = \lambda z f_\xi, \ y = \lambda z f_\eta.
\end{align}
In terms of the variables $f_\xi$ and $f_\eta$, we can express it as
\begin{align}
    \boxed{ U_2(x=\lambda zf_\xi, y=\lambda zf_\eta) = 
    \frac{e^{jkz}}{j\lambda z} e^{j\frac{k\lambda^2z}{2}(f_\xi^2+f_\eta^2)}
    \mathcal{F}\{U_1(x, y)\}. }
\end{align}

Since in FFT, the range of $f_\xi$ is from $-1/(2\Delta\xi)$ to $1/(2\Delta\xi)$ in step of $1/L$,
the range of $x$ is a scaled version of $f_\xi$, which is from $-\lambda z / (2\Delta\xi)$ to 
$\lambda z / (2\Delta \xi)$ in step of $\lambda z/L$.

At the critical sampling condition, $\lambda z/\Delta \xi = L$, which implies the observation plane
dimension equals to the source plane dimension.

\end{document}
