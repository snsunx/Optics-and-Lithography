\documentclass{article}

\usepackage{amsmath}
\usepackage{amsthm}
\usepackage{amssymb}
\usepackage{hyperref}
\usepackage{minted}
\usepackage{cleveref}

\begin{document}

\title{Signal Processing}
\author{Shi-Ning Sun}
\date{\today}
\maketitle

\section{Linear Time-Invariant Systems}

\subsection{Impulse Response}

Any discrete-time signal can be written as
\begin{align}
    x[n] = \sum_{k=-\infty}^\infty x[k]\delta[n-k].
\end{align}

If the system is linear, the output $y[n]$ can be expressed as
\begin{align}
    y[n] = \sum_{k=-\infty}^\infty x[k] h_k[n]
\end{align}
where $h_k[n]$ is the response of $\delta[n-k]$.

If the system is further time-invariant, we can write $h_k[n] = h[n - k]$, and the output is
\begin{align}
    y[n] = \sum_{k=-\infty}^\infty x[k] h[n-k].
\end{align}

The function $h[n]$ is called impulse response and completely characterizes the behavior of 
a linear time-invariant (LTI) system.

\subsection{Complex Exponentials}

The more common way to handle LTI systems is to represent the signals as a linear combinations
of complex exponentials instead of impulses. The reason is that complex exponentials are 
eigenfunctions of LTI systems, which implies they only pick up an amplitude change after
transforming through the system. 

To illustrate this, consider transforming $x(t) = e^{st}$ through an LTI system
\begin{align}
    y(t) = \int_{-\infty}^\infty h(\tau) x(t-\tau) d\tau = \int_{-\infty}^\infty h(\tau) e^{s(t-\tau)} d\tau
    = e^{st} \underbrace{\int_{-\infty}^\infty h(\tau) e^{-s\tau}}_{H(s)} d\tau.
\end{align}
Similarly, for a discrete-time signal $x[n] = z^n$,
\begin{align}
    y[n] = \sum_{k=-\infty}^\infty h[k] x[n-k] = \sum_{k=-\infty}^\infty h[k] z^{n-k} = z^n 
    \underbrace{\sum_{k=-\infty}^\infty h[k]z^{-k}}_{H(z)}
\end{align}
This transform of the impulse resposne is called the transfer function.

The usefulness of the eigenfunction property of complex exponentials is that, if we represent
an input signal as a linear combination of complex exponentials, each term in the linear combination
only picks up an amplitude proportional to the transfer function. Specifically, in continuous time
\begin{align}
    x(t) = \sum_k a_k e^{s_kt} \longmapsto y(t) = \sum_k a_k H(s_k) e^{s_kt}
\end{align}
and in discrete time
\begin{align}
    x[n] = \sum_k a_k z_k^n \longmapsto y[n] = \sum_k a_k H(z) z_k^n. 
\end{align}

Typically, we will let $s = j\omega$ and $z = e^{j\omega}$, in which case the transform is the Fourier transform in continuous or discrete time. In such case we write
\begin{align}
    H(\omega) \equiv H(s = j\omega) = \int_{-\infty}^\infty h(t) e^{-j\omega t} dt.
\end{align}
and
\begin{align}
    H(\omega) = \equiv H(z = e^{j\omega}) = \sum_{n=\infty}^\infty h[n] e^{-j\omega n}
\end{align}


\section{Fourier Series and Transform}

Periodic signals can be represented as Fourier series, which is a finite sum of complex exponentials.
Aperiodic signals can also be represented as a sum but the sum is infinite.

\subsection{Continuous-Time Fourier Series}

A continuous-time signal is periodic if for all $t$
\begin{align}
    x(t) = x(t + T).
\end{align}

The Fourier series representation of such a signal is
\begin{align}
    a_k &= \frac{1}{T} \int_T x(t) e^{-jk\omega_0t}dt, \label{eq:ctfs_ft}\\
    x(t) &= \sum_{k=-\infty}^\infty a_k e^{jk\omega_0t}. \label{eq:ctfs_ift}
\end{align}
where $\omega_0 = 2\pi/T$.

\subsection{Discrete-Time Fourier Series}

A discrete-time signal is periodic if for all $n$
\begin{align}
    x[n] = x[n + N].
\end{align}

The Fourier series representation of such a signal is
\begin{align}
    a_k &= \frac{1}{N} \sum_{n=\langle N\rangle} x[n] e^{-jk(2\pi/N)n}, \\
    x[n] &= \sum_{k=\langle N\rangle} a_k e^{jk(2\pi/N)n}.
\end{align}


\subsection{Continuous-Time Fourier Transform}

We will write an aperiodic continuous-time signal $x(t)$ as a linear combination of complex exponentials such that

\begin{align}
    X(\omega) &= \int_{-\infty}^\infty x(t)e^{-j\omega t}dt, \\
    x(t) &= \frac{1}{2\pi}\int_{-\infty}^\infty X(\omega) e^{j\omega t}d\omega.
\end{align}

\subsection{Discrete-Time Fourier Transform}

For a discrete-time signal, 
\begin{align}
    X(\omega) &= \sum_{n=-\infty}^\infty x[n] e^{-j\omega n},\\
    x[n] &= \frac{1}{2\pi}\int_{2\pi} X(\omega) e^{j\omega n} d\omega.
\end{align}

Note that the frequency-domain representation $X(\omega)$ is periodic in $2\pi$. This explains why in the inverse FT we only need to integrate over one period of $2\pi$.

\subsection{Expressions Using Frequency}

In optics and other fields, we use frequency $f$ instead of angular frequency $\omega$. In this case, the continuous-time Fourier transform is
\begin{align}
    X(f) &= \int_{-\infty}^\infty x(t)e^{-j 2\pi f t}dt,\\
    x(t) &= \int_{-\infty}^\infty X(\omega) e^{j 2\pi f t} df.
\end{align}
and in discrete time
\begin{align}
    X(f) &= \sum_{n=-\infty}^\infty x[n] e^{-j2\pi f n},\\
    x[n] &= \int_1 X(f) e^{j2\pi f n} df.
\end{align}

\section{Fast Fourier Transform}

\subsection{Definition}

In numerical calculations, we cannot have infinite data points on the input signals, which means 
the input signal will need to be restricted both in total time duration and intervals between the 
individual data points. The discrete Fourier transform is essentially the Fourier series
representation of discrete-time periodic signals with some modifications:
\begin{align}
    y_k = \sum_{m=0}^{n-1} x_m e^{-j2\pi km/n}, \label{eq:y_k} \\
    x_m = \frac{1}{n} \sum_{k=0}^{n-1} y_k e^{j2\pi km/n}. \label{eq:x_m}
\end{align}
This is the same as the definition of Fourier transform from discrete-time Fourier series except by
moving the coefficient $1/n$ from the inverse FT to FT.

Another way to look at the discrete Fourier transform expressions is by regarding it as a modification
of the continuous-time Fourier series. Toward that end, we need to redefine the continuous-time Fourier
series expressions Eqs.~\cref{eq:ctfs_ft,eq:ctfs_ift} to be
\begin{align}
    y_k &= \int_T x(t) e^{-j2\pi kt/T}dt, \label{eq:fft_ctfs_ft} \\
    x(t) &= \frac{1}{T} \sum_{k=-\infty}^\infty y_k e^{j2\pi kt/T}. \label{eq:fft_ctfs_ift}
\end{align}
Suppose $x(t)$ is sampled at $m\Delta t$, with $m = 0, ..., n-1$. We can write Eq.~\ref{eq:fft_ctfs_ft}
as a Riemann sum,
\begin{align}
    y_k &= \sum_{m=0}^{n-1} x(m\Delta t) e^{-j2\pi km\Delta t / (n\Delta t)} \Delta t\\
        &= \underbrace{\Delta t}_{{\tt dt}} \underbrace{\sum_{m=0}^{n-1} x_m e^{-j2\pi km / n}}_{{\tt fft(x)}} \\
\end{align}
Therefore, to accurately represent the frequency-domain signal, we need to multiply the output from
the FFT function {\tt fft(x)} by a correction factor {\tt dt}.

Similarly, on the inverse Fourier transform,
\begin{align}
    x_m \equiv x(m\Delta t) &= \frac{1}{T} \sum_{k=-\infty}^\infty y_k e^{j2\pi kt/T}\\
    &= \underbrace{\frac{1}{\Delta t}}_{{\tt 1/dt}}
       \underbrace{\left(\frac{1}{n}\sum_{k=0}^{n-1} y_k e^{j2\pi km\Delta t/(n\Delta t)}\right)}_{{\tt ifft(x)}}
\end{align}
Hence to accurately represent the time-domain signal, we need to multiply the output from the {\tt ifft}
function by a correction factor {\tt 1/dt}.

\subsection{Sampling}

A practical issue that arises in numerical calculation is that whether the zero point is at the
start or in the center of the time or frequency variable range. If we sample the time domain from
$t = 0$, the samples are at times
\begin{align}
    0, \ \Delta t, \ 2\Delta t, \ ..., \ (n-1)\Delta t. \label{eq:normal_order}
\end{align}
For other applications, particularly in optics where the variables are in the spatial domain,
the samples are more likely to locate at (for simplicity we will assume $n$ is even)
\begin{align}
    (-n/2)\Delta t, \ (-n/2+1)\Delta t,\ ...,\ 0,\ \Delta t,\ ...,\ (n/2-1)\Delta t.
\end{align}
In such cases, we need to first convert the samples to the order in Eq.~\ref{eq:normal_order}
before passing to the FFT function on a computer.

Similarly, when we get back the returned array of $y_k$ values from the computer, they are in
the order of
\begin{align}
    0, \ \Delta f, \ 2\Delta f, ... \ (n-1)\Delta f.
\end{align}
We either want to keep them as they are, or in the case of spatial-domain applications, we need
to convert them to
\begin{align}
    (-n/2)\Delta f, \ (-n/2+1)\Delta f,\ ...,\ 0,\ \Delta f,\ ...,\ (n/2-1)\Delta f.
\end{align}

\subsection{Code}

In both Python and MATLAB, the functions to perform FFT are very similar. We will first describe
the code in Python before getting to MATLAB.

\subsubsection{Python}

In Python, FFT can be performed with the {\tt np.fft} module ({\tt numpy} is imported as {\tt np}).
The {\tt np.fft} and {\tt np.ifft} functions exactly convert between $x_m$ and $y_k$ defined in
Eq.~\ref{eq:x_m} and Eq.~\ref{eq:y_k}. To resolve the sampling starting point issue, we have two
solutions: 1. Use {\tt np.fft.fftfreq} as the x-axis when plotting, which returns an {\tt np.ndarray}
with zero-centered coordinates. 2. Use {\tt np.fft.fftshift} on the output to make the output zero-
centered. This function can also be used to convert the input from zero-centered to zero-starting.

Suppose we prepared the time points and the time-domain input as follows
\begin{minted}{python}
    import numpy as np

    T = <total-time>
    dt = <time-interval>
    n = T // dt # assuming n is even

    t = np.arange(0, t, dt)
    x = <commands-to-obtain-input>
\end{minted}

If we want to zero-center the output, we can directly obtain the frequencies from {\tt np.fft.fftfreq} by
\begin{minted}{python}
    f = np.fft.fftfreq(n, dt)
    y = np.fft.fft(x)
\end{minted}
or if the frequencies are alreay prepared in the normal order, we can do {\tt np.fft.fftshift} on the
output as
\begin{minted}{python}
    f = np.arange(-1/(2*dt), 1/(2*dt), 1/T)
    y = np.fft.fftshift(np.fft.fft(x))
\end{minted}

If we don't want to zero-center the output, we can simply create the frequencies as follows and not modify
the output
\begin{minted}{python}
    f = np.arange(0, 1/dt, 1/T)
    y = np.fft.fft(x)
\end{minted}

\subsubsection{MATLAB}

In MATLAB, the flow of the code is the same:
\begin{minted}{matlab}
    T = <total-time>
    dt = <time-interval>

    t = 0:dt:T-dt
    x = <commands-to-obtain-input>
\end{minted}

MATLAB does not have the {\tt fftfreq} function. If we want to zero-center the output, we can only use
the {\tt fftshift} function
\begin{minted}{matlab}
    f = -1/(2*dt):1/T:1/(2*dt)-1/T
    y = fftshift(fft(x))
\end{minted}

If we don't want to zero-center the output, we can simply create the frequencies as follows and not modify
the output
\begin{minted}{matlab}
    f = 0:1/T:1/dt
    y = fft(x)
\end{minted}

\end{document}
